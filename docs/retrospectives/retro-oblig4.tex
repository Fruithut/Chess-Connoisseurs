\documentclass{article}

\title{Retrospective for oblig 4}
\author{Group 5}
\date{}

\begin{document}
    \maketitle

    \section{Thoughts}
    \noindent
     We continued like the last oblig to hold meetings twice a week. We had some days where we booked rooms for over 2 hours (6 hours) to work on the project.
     during the meetings we worked in groups focusing on different tasks. One group worked on the SQL database and its implementasion, another group worked on sound effects
     and the last group worked on fixing bugs and improving the AI. We continued using Trello to track the workprogress of the different groups and what they had accomplished.
     We also talked about future implementasions and ideas for the project such as different gamemodes.
     Implementing the SQL files and database was more difficult than expected.
     The overall work amount was better suited to the time we had to finish everything, compared to earlier itterations.





     //Last oblig retro
      The meetings were used to inform and explain what we had
     done individually. Doing this we had more control over what the others had accomplished and how we could use it to work together.
     We used Trello to track what needed to be done, what was still being worked on and what was done. Using Trello was a good choice, since we could mark what task we
     wanted to do, until when it had to be done and who was allocated to which tasks. We found out that Programming the AI from scratch was harder than expected
     and that we needed more resources for it. Not only was the AI hard, there was a lot of work to be done in a short timespan
     resulting in some hectic periods. Everybody in the team has started to use git, and know how to handle error and merge problems.
     Everybody got to touch the code in some way, and contribute to the project in their own way. In the programming part of the project
     we used javafx and inspiration from other open source codes(see README.md). In the next oblig we want to
     improve our GUI with more functions, for example better login or other functions. We also maybe want to implement sound to the game, like when a player moves a piece it makes a *click* sound or background music. We will continue to use trello and some of our roles that we have at the moment might change during the next oblig.

\end{document}
